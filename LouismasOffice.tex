% !TEX program = LuaLaTeX+se

\documentclass[11pt]{book}

\usepackage{fontspec}
\setmainfont[Ligatures=Rare, Numbers=Proportional, Numbers=OldStyle, Scale=1.1]{Crimson Text}
\title{\textbf{\huge{Officium S. Ludovici Regis Franciæ}}}
\date{XI Maii MMXXV}
\newfontfamily\myfont{EB Garamond}

\usepackage{lettrine}
\usepackage{multicol}
\usepackage[autocompile]{gregoriotex}

\begin{document}

\begin{center}
\textsc{\huge{In Festo S. Ludovici Regis Franciæ}}

\vspace{0.5cm}
\textsc{\large{AD I. VESPERAS}}
\end{center}

\vspace{0.25cm}
Pater noster. Ave Maria.

\grechangestafflinethickness{35}
\gresetlinecolor{gregoriocolor}
\vspace{0.25cm}
\gregorioscore{deusinadiutorium}
\vspace{0.25cm}
\gregorioscore{firstvespersant1}
\vspace{0.25cm}
\begin{multicols}{2}
2. Donec ponam ini\textbf{mí}cos \textbf{tu}os, * scabéllum pe\textit{dum tu}\textbf{ó}rum.

3. Virgam virtútis tuæ emíttet Dómi\textbf{nus} ex \textbf{Si}on: * domináre in médio inimicó\textit{rum tu}\textbf{ó}rum.

4. Tecum princípium in die virtútis tuæ in splendóri\textbf{bus} sanc\textbf{tó}rum: * ex útero ante lucíferum \textit{génu}\textbf{i} te.

5. Iurávit Dóminus, et non pœni\textbf{té}bit \textbf{e}um: * Tu es sacérdos in ætérnum secúndum órdi\textit{nem Mel}\textbf{chí}sedech.

6. Dóminus a \textbf{dex}tris \textbf{tu}is, * confrégit in die iræ \textit{suæ} \textbf{re}ges.

7. Iudicábit in natiónibus, imp\textbf{lé}bit ru\textbf{í}nas: * conquassábit cápita in ter\textit{ra mul}\textbf{tó}rum.

8. De torrénte in \textbf{vi}a \textbf{bi}bet: * proptérea exal\textit{tábit} \textbf{ca}put.

9. Glória \textbf{Pa}tri et \textbf{Fí}lio * et Spirí\textit{tui} \textbf{Sanc}to.

10. Sicut erat in princípio et \textbf{nunc} et \textbf{sem}per * et in sǽcula sǽcu\textit{lórum.} \textbf{A}men.

\textit{Repetitur Antiphona.}
\end{multicols}

\vspace{0.25cm}
\gregorioscore{firstvespersant2}
\vspace{0.25cm}
\begin{multicols}{2}
2. Magna ópera \textbf{Dó}mini: * exquisíta in omnes voluntá\textit{tes} \textbf{e}ius.

3. Conféssio et magnificéntia opus \textbf{e}ius: * et iustítia eius manet in sǽcu\textit{lum} \textbf{sǽ}culi.

4. Memóriam fecit mirabílium suórum, miséricors et miserátor \textbf{Dó}minus: * escam dedit timén\textit{ti}\textbf{bus} se.

5. Memor erit in sǽculum testaménti \textbf{su}i: * virtútem óperum suórum annuntiábit pópu\textit{lo} \textbf{su}o:

6. Ut det illis hereditátem \textbf{gén}tium: * ópera mánuum eius véritas, et \textit{iu}\textbf{dí}cium.

7. Fidélia ómnia mandáta eius: † confirmáta in sǽculum \textbf{sǽ}culi, * facta in veritáte et æ\textit{qui}\textbf{tá}te.

8. Redemptiónem misit pópulo \textbf{su}o: * mandávit in ætérnum testamén\textit{tum} \textbf{su}um.

9. \textit{(fit reverentia)} Sanctum, et terríbile nomen \textbf{e}jus: * inítium sapiéntiæ ti\textit{mor} \textbf{Dó}mini.

10. Intelléctus bonus ómnibus faciéntibus \textbf{e}um: * laudátio ejus manet in sǽcu\textit{lum} \textbf{sǽ}culi.

11. Glória Patri et \textbf{Fí}lio * et Spirítu\textit{i} \textbf{Sanc}to.

12. Sicut erat in princípio et nunc et \textbf{sem}per * et in sǽcula sǽculó\textit{rum.} \textbf{A}men.
\end{multicols}

\vspace{0.25cm}
\gregorioscore{firstvespersant3}
\vspace{0.25cm}
\begin{multicols}{2}
2. Potens in terra erit \textbf{se}men \textbf{e}jus: * generátio rectórum be\textit{nedi}\textbf{cé}tur.

3. Glória, et divítiæ in \textbf{do}mo \textbf{e}jus: * et iustítia ejus manet in sǽcu\textit{lum} \textbf{sǽ}culi.

4. Exórtum est in ténebris \textbf{lu}men \textbf{rec}tis: * miséricors, et miserá\textit{tor,} \textbf{et} iustus.

5. Iucúndus homo qui miserétur et cómmodat, † dispónet sermónes suos \textbf{in} iu\textbf{dí}cio: * quia in ætérnum non \textit{commo}\textbf{vé}bitur.

6. In memória ætérna \textbf{e}rit \textbf{iu}stus: * ab auditióne mala \textit{non ti}\textbf{mé}bit.

7. Parátum cor eius speráre in Dómino, † confirmátum \textbf{est} cor \textbf{e}ius: * non commovébitur donec despíciat ini\textit{mícos} \textbf{su}os.

8. Dispérsit, dedit paupéribus: † iustítia ejus manet in \textbf{sǽ}culum \textbf{sǽ}culi, * cornu ejus exaltábi\textit{tur in} \textbf{gló}ria.

9. Peccátor vidébit, et irascétur, † déntibus suis fremet \textbf{et} ta\textbf{bé}scet: * desidérium peccató\textit{rum pe}\textbf{rí}bit.

10. Glória \textbf{Pa}tri et \textbf{Fí}lio * et Spirí\textit{tui} \textbf{Sanc}to.

11. Sicut erat in princípio et \textbf{nunc} et \textbf{sem}per * et in sǽcula sǽcu\textit{lórum.} \textbf{A}men.
\end{multicols}

\vspace{0.25cm}
\gregorioscore{firstvespersant4}
\vspace{0.25cm}
\begin{multicols}{2}
2. \textit{(fit reverentia)} Sit nomen Dómini \textit{bene}\textbf{díc}tum, * ex hoc nunc, et \textit{usque in} \textbf{sǽ}culum.

3. A solis ortu usque \textit{ad oc}\textbf{cá}sum, * laudábi\textit{le nomen} \textbf{Dó}mini.

4. Excélsus super omnes \textit{gentes} \textbf{Dó}minus, * et super cælos \textit{glória} \textbf{e}jus.

5. Quis sicut Dóminus, Deus noster, qui in \textit{altis} \textbf{há}bitat, * et humília réspicit in cæ\textit{lo et in} \textbf{ter}ra?

6. Súscitans a \textit{terra} \textbf{í}nopem, * et de stércore \textit{érigens} \textbf{páu}perem:

7. Ut cóllocet eum \textit{cum prin}\textbf{cí}pibus, * cum princípibus \textit{pópuli} \textbf{su}i.

8. Qui habitáre facit stéri\textit{lem in} \textbf{do}mo, * matrem fili\textit{órum læ}\textbf{tán}tem.

9. Glória Pa\textit{tri et} \textbf{Fí}lio * et Spi\textit{rítui} \textbf{Sanc}to.

10. Sicut erat in princípio et \textit{nunc et} \textbf{sem}per * et in sǽcula sǽ\textit{culórum.} \textbf{A}men.
\end{multicols}

\vspace{0.25cm}
\gregorioscore{firstvespersant5}
\vspace{0.25cm}
\begin{multicols}{2}
2. Quóniam confirmáta est super nos misericórdia \textbf{e}jus: * et véritas Dómini manet \textbf{in} æ\textbf{tér}num.

3. Glória Patri et \textbf{Fí}lio * et Spi\textbf{rí}tui \textbf{Sanc}to.

4. Sicut erat in princípio et nunc et \textbf{sem}per * et in sǽcula sǽcu\textbf{ló}rum. \textbf{A}men.
\end{multicols}

\begin{center}
\textit{Capitulum. Cf. Es 7:27}
\end{center}

\begin{multicols}{2}
\lettrine{B}{enedictus} Dominus Deus patrem nostrorum † qui dedit hanc voluntatem in cor regis clarificare domum suam * quæ est in Jerusalem.

\begin{myfont}℟.\end{myfont} Deo gratias.
\end{multicols}

\vspace{0.25cm}
\gregorioscore{firstvespershymn}
\vspace{0.25cm}
\begin{myfont}℣.\end{myfont} David regis sedet in solio.

\begin{myfont}℟.\end{myfont} Salomonis utens iudicio.

\vspace{0.25cm}
\gregorioscore{firstvespersmagant}
\vspace{0.25cm}
\begin{multicols}{2}
3. Quia respéxit humilitátem ancíl\textit{læ} \textbf{su}æ: * ecce enim ex hoc beátam me dicent omnes gene\textit{rati}\textbf{ó}nes.

4. Quia fecit mihi magna \textit{qui} \textbf{po}tens est: * et sanctum \textit{nomen} \textbf{e}ius.

5. Et misericórdia eius, a progénie in \textit{pro}\textbf{gé}nies: * timén\textit{tibus} \textbf{e}um.

6. Fecit poténtiam in brácchi\textit{o} \textbf{su}o: * dispérsit supérbos mente \textit{cordis} \textbf{su}i.

7. Depósuit poténtes \textit{de} \textbf{se}de: * et exal\textit{távit} \textbf{hú}miles.

8. Esuriéntes implé\textit{vit} \textbf{bo}nis: * et dívites dimí\textit{sit in}\textbf{á}nes.

9. Suscépit Israël púe\textit{rum} \textbf{su}um: * recordátus misericór\textit{diæ} \textbf{su}æ.

10. Sicut locútus est ad pa\textit{tres} \textbf{no}stros: * Ábraham, et sémini ei\textit{us in} \textbf{sǽ}cula.

11. Glória Patri, \textit{et} \textbf{Fí}lio, * et Spirí\textit{tui} \textbf{Sanc}to.

12. Sicut erat in princípio, et nunc, \textit{et} \textbf{sem}per, * et in sǽcula sæcu\textit{lórum.} \textbf{A}men.
\end{multicols}

\begin{myfont}℣.\end{myfont} Dominus vobiscum.

\begin{myfont}℟.\end{myfont} Et cum spiritu tuo.

Oremus.

\begin{center}
\textit{Oratio:}
\end{center}

\begin{multicols}{2}
\lettrine{D}{eus} qui beatum Ludovicum confessorem tuum de terreno ac temporali regno ad caelestis et aeterni gloriam transtulisti; † eius quæsumus meritis et intercessione regis regum Jesu Christi, Filii Tui, * nos coheredes efficias et eiusdem regni tribuas esse consortes. Per Dóminum nostrum Iesum Christum, Fílium tuum: qui tecum vivit et regnat in unitáte Spíritus Sancti, Deus, per ómnia sǽcula sæculórum.

\begin{myfont}℟.\end{myfont} Amen.
\end{multicols}

\begin{myfont}℣.\end{myfont} Dominus vobiscum.

\begin{myfont}℟.\end{myfont} Et cum spiritu tuo.

\vspace{0.25cm}
\gregorioscore{firstvespersbenedicamus}
\vspace{0.25cm}

\begin{myfont}℣.\end{myfont} Fidelium animæ per misericordiam Dei requiescant in pace.

\begin{myfont}℟.\end{myfont} Amen.

\begin{center}
\large{AD COMPLETORIVM}
\end{center}

\textit{Lector incipit}
\begin{myfont}℣.\end{myfont} Iube Domne, benedicere.

\textit{Benedictio:} Noctem quietam et finem perfectum concedat nobis Dominus omnipotens.

\begin{myfont}℟.\end{myfont} Amen.

\begin{center}
\textit{Lectio brevis. 1. Pet.5.c.}
\end{center}

\begin{multicols}{2}
\lettrine{F}{ratres:} sobrii estote et vigilate: † qui adversarius vester diabolus tamquam leo rugiens circuit, quærens quem devoret: * cui resistite fortes in fide. Tu autem Domine, miserere nobis.
\begin{myfont}℟.\end{myfont} Deo gratias.
\end{multicols}

\begin{myfont}℣.\end{myfont} Adiutorium nostrum in nomine Domini.

\begin{myfont}℟.\end{myfont} Qui fecit cælum et terram.

Pater noster. \textit{secreto.}

\textit{Deinde sacerdos dicit:} 

\lettrine{C}{onfiteor} Deo omnipotenti, beatæ Mariæ semper Virgini, beato Michaelo Archangelo, beato Iohanni Baptistæ, sanctis Apostolis Petro et Paulo, omnibus sanctis et vobis, fratres, quia peccavi nimis cogitatione, verbo, et opere: mea culpa, mea culpa, mea maxima culpa. Ideo precor beatam Mariam semper Virginem, beatum Michaelum Archangelum, beatum Iohannem Baptistam, sanctos Apostolos Petrum et Paulum, omnes sanctos, et vos fratres, orare pro me ad Dominum Deum nostrum.

\vspace{0.25cm}
\textit{Et chorus respondit}

\lettrine{M}{isereatur} tui omnipotens Deus, et dimissis peccatis tuis, perducat te ad vitam æternam.
\begin{myfont}℟.\end{myfont} Amen.

\vspace{0.25cm}
\textit{Et chorus repetit:}

\lettrine{C}{onfiteor} Deo omnipotenti, beatæ Mariæ semper Virgini, beato Michaelo Archangelo, beato Iohanni Baptistæ, sanctis Apostolis Petro et Paulo, omnibus sanctis et tibi, pater, quia peccavi nimis cogitatione, verbo, et opere: mea culpa, mea culpa, mea maxima culpa. Ideo precor beatam Mariam semper Virginem, beatum Michaelum Archangelum, beatum Iohannem Baptistam, sanctos Apostolos Petrum et Paulum, omnes sanctos, et te, pater, orare pro me ad Dominum Deum nostrum.

\vspace{0.25cm}
\textit{Et sacerdos respondit:}

\lettrine{M}{isereatur} vestri omnipotens Deus, et dimissis peccatis vestris, perducat vos ad vitam æternam.
\begin{myfont}℟.\end{myfont} Amen.
\vspace{0.25cm}
\lettrine{I}{ndulgentiam} absolutionem et remissionem peccatorum nostrorum tribuat vobis omnipotens et misericors Dominus.
\begin{myfont}℟.\end{myfont} Amen.

\vspace{0.5cm}
\gresetinitiallines{0}
\gregorioscore{complineconverte}
\vspace{0.25cm}
\gresetinitiallines{1}
\gregorioscore{complinepsalmsant}
\begin{multicols}{2}
2. Miserére mei, * et exáudi oratiónem meam.

3. Fílii hóminum, úsquequo gravi corde? * ut quid dilígitis vanitátem, et quǽritis mendácium?

4. Et scitóte quóniam mirificávit Dóminus sanctum suum: * Dóminus exáudiet me cum clamávero ad eum.

5. Irascímini, et nolíte peccáre: † quæ dícitis in córdibus vestris, * in cubílibus vestris compungímini.

6. Sacrificáte sacrifícium justítiæ, † et speráte in Dómino. * Multi dicunt: Quis osténdit nobis bona?

7. Signátum est super nos lumen vultus tui, Dómine: * dedísti lætítiam in corde meo.

8. A fructu fruménti, vini, et ólei sui * multiplicáti sunt.

9. In pace in idípsum * dórmiam, et requiéscam;

10. Quóniam tu, Dómine, singuláriter in spe * constituísti me.

11. Glória Patri et \textbf{Fí}lio * et Spi\textbf{rí}tui \textbf{Sanc}to.

12. Sicut erat in princípio et nunc et \textbf{sem}per * et in sǽcula sǽcu\textbf{ló}rum. \textbf{A}men.

\vspace{0.25cm}
\lettrine{Q}{ui} hábitat in adjutório Altíssimi, * in protectióne Dei cæli commorábitur.

2. Dicet Dómino: Suscéptor meus es tu, et refúgium meum: * Deus meus sperábo in eum.

3. Quóniam ipse liberávit me de láqueo venántium, * et a verbo áspero.

4. Scápulis suis obumbrábit tibi: * et sub pennis ejus sperábis.

5. Scuto circúmdabit te véritas ejus: * non timébis a timóre noctúrno,

6. A sagítta volánte in die, a negótio perambulánte in ténebris: * ab incúrsu, et dæmónio meridiáno.

7. Cadent a látere tuo mille, et decem míllia a dextris tuis: * ad te autem non appropinquábit.

8. Verúmtamen óculis tuis considerábis: * et retributiónem peccatórum vidébis.

9. Quóniam tu es, Dómine, spes mea: * Altíssimum posuísti refúgium tuum.

10. Non accédet ad te malum: * et flagéllum non appropinquábit tabernáculo tuo.

11. Quóniam Ángelis suis mandávit de te: * ut custódiant te in ómnibus viis tuis.

12. In mánibus portábunt te: * ne forte offéndas ad lápidem pedem tuum.

13. Super áspidem, et basilíscum ambulábis: * et conculcábis leónem et dracónem.

14. Quóniam in me sperávit, liberábo eum: * prótegam eum, quóniam cognóvit nomen meum.

15. Clamábit ad me, et ego exáudiam eum: * cum ipso sum in tribulatióne: erípiam eum et glorificábo eum.

16. Longitúdine diérum replébo eum: * et osténdam illi salutáre meum.

17. Glória Patri et \textbf{Fí}lio * et Spi\textbf{rí}tui \textbf{Sanc}to.

18. Sicut erat in princípio et nunc et \textbf{sem}per * et in sǽcula sǽcu\textbf{ló}rum. \textbf{A}men.

\vspace{0.25cm}
\lettrine{E}{cce} nunc benedícite Dóminum, * omnes servi Dómini:

2. Qui statis in domo Dómini, * in átriis domus Dei nostri.

3. In nóctibus extóllite manus vestras in sancta, * et benedícite Dóminum.

4. Benedícat te Dóminus ex Sion, * qui fecit cælum et terram.

5. Glória Patri, et \textbf{Fí}lio, * et Spi\textbf{rí}tui \textbf{Sanc}to.

6. Sicut erat in princípio et nunc et \textbf{sem}per * et in sǽcula sǽcu\textbf{ló}rum. \textbf{A}men.
\end{multicols}
\gregorioscore{complinepsalmsant}
\vspace{0.25cm}
[Hymnus \textit{Te lucis ante terminum}]

\begin{center}
\textit{Capitulum Ier. 14.}
\end{center}
\begin{multicols}{2}
\lettrine{T}{u} autem in nobis es Domine: † et nomen sanctum tuum invocatum est super nos: * ne derelinquas nos Domine Deus noster.

\begin{myfont}℟.\end{myfont} Deo gratias.
\end{multicols}

\gregorioscore{complineshortresp}
\vspace{0.25cm}
\begin{myfont}℣.\end{myfont} Custodi nos Domine ut pupillam oculi.

\begin{myfont}℟.\end{myfont} Sub umbra alarum tuarum protege nos.
\vspace{0.25cm}
\gregorioscore{complinenuncdimittis}
\begin{multicols}{2}
2. Quia vidérunt ócu\textit{li} \textbf{me}i * salu\textit{táre} \textbf{tu}um:

3. Quod \textit{pa}\textbf{rá}sti * ante fáciem ómnium \textit{popu}\textbf{ló}rum:

4. Lumen ad revelatió\textit{nem} \textbf{gén}tium, * et glóriam plebis \textit{tuæ} \textbf{Ís}rael.

5. Glória Patri, \textit{et} \textbf{Fí}lio, * et Spirí\textit{tui} \textbf{Sanc}to.

6. Sicut erat in princípio, et nunc, \textit{et} \textbf{sem}per, * et in sǽcula sǽcu\textit{lórum.} \textbf{A}men.
\end{multicols}

\begin{myfont}℣.\end{myfont} Dominus vobiscum.

\begin{myfont}℟.\end{myfont} Et cum spiritu tuo.

Oremus.

\begin{center}
\textit{Oratio:}
\end{center}

\begin{multicols}{2}
\lettrine{V}{isita,} quǽsumus, Dómine, habitatiónem istam, et omnes insídias inimíci ab ea lónge repélle: † Ángeli tui sancti hábitent in ea, qui nos in pace custódiant; * et benedíctio tua sit super nos semper. Per Dóminum nostrum Iesum Christum, Fílium tuum: qui tecum vivit et regnat in unitáte Spíritus Sancti, Deus, per ómnia sǽcula sæculórum.

\begin{myfont}℟.\end{myfont} Amen.
\end{multicols}
\begin{myfont}℣.\end{myfont} Dominus vobiscum.

\begin{myfont}℟.\end{myfont} Et cum spiritu tuo.

\begin{myfont}℣.\end{myfont} Benedicamus Domino.

\begin{myfont}℟.\end{myfont} Deo gratias.

\textit{Benedictio:} Benedicat et custodiat nos omnipotens et misericors Dominus, Pater, et Filius, et Spiritus Sanctus.

\begin{myfont}℟.\end{myfont} Amen.

\begin{center}
\large{AD MATVTINVM}
\end{center}

\vspace{0.25cm}
Pater noster. Ave Maria. Credo. Reliqua matutinorum.

\begin{center}
\large{AD LAVDES}
\end{center}

\vspace{0.25cm}
Pater noster. Ave Maria. Credo.

Deus in adiutorium.

\gregorioscore{laudsant1}
\begin{multicols}{2}
2. Étenim firmávit orbem terræ, * qui non commovébitur.

3. Paráta sedes tua ex tunc: * a sǽculo tu es.

4. Elevavérunt flúmina, Dómine: * elevavérunt flúmina vocem suam.

5. Elevavérunt flúmina fluctus suos, * a vócibus aquárum multárum.

6. Mirábiles elatiónes maris: * mirábilis in altis Dóminus.

7. Testimónia tua credibília facta sunt nimis: * domum tuam decet sanctitúdo, Dómine, in longitúdinem diérum.

8. Glória Patri, et Fílio, * et Spirítui Sancto.

9. Sicut erat in princípio et nunc et semper * et in sǽcula sǽculórum. Amen.
\end{multicols}
\vspace{0.25cm}
\gregorioscore{laudsant2}
\begin{multicols}{2}
2. Introíte in conspéctu ejus, * in exsultatióne.

3. Scitóte quóniam Dóminus ipse est Deus: * ipse fecit nos, et non ipsi nos.

4. Pópulus ejus, et oves páscuæ ejus: * introíte portas ejus in confessióne, átria ejus in hymnis: confitémini illi.

5. Laudáte nomen ejus: quóniam suávis est Dóminus, in ætérnum misericórdia ejus, * et usque in generatiónem et generatiónem véritas ejus.

6. Glória Patri, et Fílio, * et Spirítui Sancto.

7. Sicut erat in princípio et nunc et semper * et in sǽcula sǽculórum. Amen.
\end{multicols}
\vspace{0.25cm}
\gregorioscore{laudsant3}
\begin{multicols}{2}
2. Sitívit in te ánima mea, * quam multiplíciter tibi caro mea.

3. In terra desérta, et ínvia, et inaquósa: * sic in sancto appárui tibi, ut vidérem virtútem tuam, et glóriam tuam.

4. Quóniam mélior est misericórdia tua super vitas: * lábia mea laudábunt te.

5. Sic benedícam te in vita mea: * et in nómine tuo levábo manus meas.

6. Sicut ádipe et pinguédine repleátur ánima mea: * et lábiis exsultatiónis laudábit os meum.

7. Si memor fui tui super stratum meum, in matutínis meditábor in te: * quia fuísti adjútor meus.

8. Et in velaménto alárum tuárum exsultábo, adhǽsit ánima mea post te: * me suscépit déxtera tua.

9. Ipsi vero in vanum quæsiérunt ánimam meam, introíbunt in inferióra terræ: * tradéntur in manus gládii, partes vúlpium erunt.

10. Rex vero lætábitur in Deo, laudabúntur omnes qui jurant in eo: * quia obstrúctum est os loquéntium iníqua.

\lettrine{D}{eus} misereátur nostri, et benedícat nobis: * illúminet vultum suum super nos, et misereátur nostri.

2. Ut cognoscámus in terra viam tuam, * in ómnibus géntibus salutáre tuum.

3. Confiteántur tibi pópuli, Deus: * confiteántur tibi pópuli omnes.

4. Læténtur et exsúltent gentes: * quóniam júdicas pópulos in æquitáte, et gentes in terra dírigis.

5. Confiteántur tibi pópuli, Deus, confiteántur tibi pópuli omnes: * terra dedit fructum suum.

6. Benedícat nos Deus, Deus noster, benedícat nos Deus: * et métuant eum omnes fines terræ.

7. Glória Patri, et Fílio, * et Spirítui Sancto.

9. Sicut erat in princípio et nunc et semper * et in sǽcula sǽculórum. Amen.
\end{multicols}
\vspace{0.25cm}
\gregorioscore{laudsant4}
\begin{multicols}{2}
2. Benedícite, Ángeli Dómini, Dómino: * benedícite, cæli, Dómino.

3. Benedícite, aquæ omnes, quæ super cælos sunt, Dómino: * benedícite, omnes virtútes Dómini, Dómino.

4. Benedícite, sol et luna, Dómino: * benedícite, stellæ cæli, Dómino.

5. Benedícite, omnis imber et ros, Dómino: * benedícite, omnes spíritus Dei, Dómino.

6. Benedícite, ignis et æstus, Dómino: * benedícite, frigus et æstus, Dómino.

7. Benedícite, rores et pruína, Dómino: * benedícite, gelu et frigus, Dómino.

8. Benedícite, glácies et nives, Dómino: * benedícite, noctes et dies, Dómino.

9. Benedícite, lux et ténebræ, Dómino: * benedícite, fúlgura et nubes, Dómino.

10. Benedícat terra Dóminum: * laudet et superexáltet eum in sǽcula.

11. Benedícite, montes et colles, Dómino: * benedícite, univérsa germinántia in terra, Dómino.

12. Benedícite, fontes, Dómino: * benedícite, mária et flúmina, Dómino.

13. Benedícite, cete, et ómnia, quæ movéntur in aquis, Dómino: * benedícite, omnes vólucres cæli, Dómino.

14. Benedícite, omnes béstiæ et pécora, Dómino: * benedícite, fílii hóminum, Dómino.

15. Benedícat Israël Dóminum: * laudet et superexáltet eum in sǽcula.

16. Benedícite, sacerdótes Dómini, Dómino: * benedícite, servi Dómini, Dómino.

17. Benedícite, spíritus, et ánimæ justórum, Dómino: * benedícite, sancti, et húmiles corde, Dómino.

18. Benedícite, Ananía, Azaría, Mísaël, Dómino: * laudáte et superexaltáte eum in sǽcula.

19. \textit{(fit reverentia:)} Benedicámus Patrem et Fílium cum Sancto Spíritu: * laudémus et superexaltémus eum in sǽcula.

20. Benedíctus es, Dómine, in firmaménto cæli: * et laudábilis, et gloriósus, et superexaltátus in sǽcula.
\end{multicols}
\vspace{0.25cm}
\gregorioscore{laudsant5}
\begin{multicols}{2}
2. Laudáte eum, omnes Ángeli ejus: * laudáte eum, omnes virtútes ejus.

3.Laudáte eum, sol et luna: * laudáte eum, omnes stellæ et lumen.

4. Laudáte eum, cæli cælórum: * et aquæ omnes, quæ super cælos sunt, laudent nomen Dómini.

5. Quia ipse dixit, et facta sunt: * ipse mandávit, et creáta sunt.

6. Státuit ea in ætérnum, et in sǽculum sǽculi: * præcéptum pósuit, et non præteríbit.

7. Laudáte Dóminum de terra, * dracónes, et omnes abýssi.

8. Ignis, grando, nix, glácies, spíritus procellárum: * quæ fáciunt verbum ejus:

9. Montes, et omnes colles: * ligna fructífera, et omnes cedri.

10. Béstiæ, et univérsa pécora: * serpéntes, et vólucres pennátæ:

11. Reges terræ, et omnes pópuli: * príncipes, et omnes júdices terræ.

12. Júvenes, et vírgines: † senes cum junióribus laudent nomen Dómini: * quia exaltátum est nomen ejus solíus.

13. Conféssio ejus super cælum et terram: * et exaltávit cornu pópuli sui.

14. Hymnus ómnibus sanctis ejus: * fíliis Israël, pópulo appropinquánti sibi.

\lettrine{C}{antáte} Dómino cánticum novum: * laus ejus in ecclésia sanctórum.

2. Lætétur Israël in eo, qui fecit eum: * et fílii Sion exsúltent in rege suo.

3. Laudent nomen ejus in choro: * in týmpano, et psaltério psallant ei:

4. Quia beneplácitum est Dómino in pópulo suo: * et exaltábit mansuétos in salútem.

5. Exsultábunt sancti in glória: * lætabúntur in cubílibus suis.

6. Exaltatiónes Dei in gútture eórum: * et gládii ancípites in mánibus eórum.

7. Ad faciéndam vindíctam in natiónibus: * increpatiónes in pópulis.

8. Ad alligándos reges eórum in compédibus: * et nóbiles eórum in mánicis férreis.

9. Ut fáciant in eis judícium conscríptum: * glória hæc est ómnibus sanctis ejus.

\lettrine{L}{audáte} Dóminum in sanctis ejus: * laudáte eum in firmaménto virtútis ejus.

2. Laudáte eum in virtútibus ejus: * laudáte eum secúndum multitúdinem magnitúdinis ejus.

3 .Laudáte eum in sono tubæ: * laudáte eum in psaltério, et cíthara.

4. Laudáte eum in týmpano, et choro: * laudáte eum in chordis, et órgano.

5. Laudáte eum in cýmbalis benesonántibus: † laudáte eum in cýmbalis jubilatiónis: * (6) omnis spíritus laudet Dóminum.

6. Glória Patri, et Fílio, * et Spirítui Sancto.

7. Sicut erat in princípio et nunc et semper * et in sǽcula sǽculórum. Amen.
\end{multicols}

\end{document}