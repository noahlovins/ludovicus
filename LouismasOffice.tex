% !TEX program = LuaLaTeX+se

\documentclass[11pt]{book}

\usepackage{fontspec}
\setmainfont[Ligatures=Rare, Numbers=Proportional, Numbers=OldStyle, Scale=1.1]{Crimson Text}
\title{\textbf{\huge{Officium S. Ludovici Regis Franciæ}}}
\date{XI Maii MMXXV}
\newfontfamily\myfont{EB Garamond}

\usepackage{lettrine}
\usepackage{multicol}
\usepackage[autocompile]{gregoriotex}

\begin{document}

\begin{center}
\textsc{\huge{IN FESTO S. LVDOVICI REGIS FRANCIÆ}}

\vspace{0.5cm}
\textsc{\large{AD I. VESPERAS}}
\end{center}

\vspace{0.25cm}
Pater noster. Ave María.

\grechangestafflinethickness{35}
\gresetlinecolor{gregoriocolor}
\vspace{0.25cm}
\gregorioscore{deusinadiutorium}
\vspace{0.25cm}
\gregorioscore{firstvespersant1}
\vspace{0.25cm}
\begin{multicols}{2}
2. Donec ponam ini\textbf{mí}cos \textbf{tu}os, * scabéllum pe\textit{dum tu}\textbf{ó}rum.

3. Virgam virtútis tuæ emíttet Dómi\textbf{nus} ex \textbf{Si}on: * domináre in médio inimicó\textit{rum tu}\textbf{ó}rum.

4. Tecum princípium in die virtútis tuæ in splendóri\textbf{bus} sanc\textbf{tó}rum: * ex útero ante lucíferum \textit{génu}\textbf{i} te.

5. Iurávit Dóminus, et non pœni\textbf{té}bit \textbf{e}um: * Tu es sacérdos in ætérnum secúndum órdi\textit{nem Mel}\textbf{chí}sedech.

6. Dóminus a \textbf{dex}tris \textbf{tu}is, * confrégit in die iræ \textit{suæ} \textbf{re}ges.

7. Iudicábit in natiónibus, imp\textbf{lé}bit ru\textbf{í}nas: * conquassábit cápita in ter\textit{ra mul}\textbf{tó}rum.

8. De torrénte in \textbf{vi}a \textbf{bi}bet: * proptérea exal\textit{tábit} \textbf{ca}put.

9. Glória \textbf{Pa}tri et \textbf{Fí}lio * et Spirí\textit{tui} \textbf{Sanc}to.

10. Sicut erat in princípio et \textbf{nunc} et \textbf{sem}per * et in sǽcula sǽcu\textit{lórum.} \textbf{A}men.

\textit{Repetitur Antiphona.}
\end{multicols}

\vspace{0.25cm}
\gregorioscore{firstvespersant2}
\vspace{0.25cm}
\begin{multicols}{2}
2. Magna ópera \textbf{Dó}mini: * exquisíta in omnes voluntá\textit{tes} \textbf{e}ius.

3. Conféssio et magnificéntia opus \textbf{e}ius: * et iustítia eius manet in sǽcu\textit{lum} \textbf{sǽ}culi.

4. Memóriam fecit mirabílium suórum, † miséricors et miserátor \textbf{Dó}minus: * escam dedit timén\textit{ti}\textbf{bus} se.

5. Memor erit in sǽculum testaménti \textbf{su}i: * virtútem óperum suórum annuntiábit pópu\textit{lo} \textbf{su}o:

6. Ut det illis hereditátem \textbf{gén}tium: * ópera mánuum eius véritas, et \textit{iu}\textbf{dí}cium.

7. Fidélia ómnia mandáta eius: † confirmáta in sǽculum \textbf{sǽ}culi, * facta in veritáte et æ\textit{qui}\textbf{tá}te.

8. Redemptiónem misit pópulo \textbf{su}o: * mandávit in ætérnum testamén\textit{tum} \textbf{su}um.

9. \textit{(fit reverentia)} Sanctum, et terríbile nomen \textbf{e}ius: * inítium sapiéntiæ ti\textit{mor} \textbf{Dó}mini.

10. Intelléctus bonus ómnibus faciéntibus \textbf{e}um: * laudátio eius manet in sǽcu\textit{lum} \textbf{sǽ}culi.

11. Glória Patri et \textbf{Fí}lio * et Spirítu\textit{i} \textbf{Sanc}to.

12. Sicut erat in princípio et nunc et \textbf{sem}per * et in sǽcula sǽculó\textit{rum.} \textbf{A}men.
\end{multicols}

\vspace{0.25cm}
\gregorioscore{firstvespersant3}
\vspace{0.25cm}
\begin{multicols}{2}
2. Potens in terra erit \textbf{se}men \textbf{e}ius: * generátio rectórum \textbf{be}nedi\textbf{cé}tur.

3. Glória, et divítiæ in \textbf{do}mo \textbf{e}ius: * et iustítia eius manet in \textbf{sǽ}culum \textbf{sǽ}culi.

4. Exórtum est in ténebris \textbf{lu}men \textbf{rec}tis: * miséricors, et mise\textbf{rá}tor, et \textbf{ius}tus.

5. Iucúndus homo qui miserétur et cómmodat, † dispónet sermónes suos \textbf{in} iu\textbf{dí}cio: * quia in ætérnum non \textbf{com}mo\textbf{vé}bitur.

6. In memória ætérna \textbf{e}rit \textbf{iu}stus: * ab auditióne mala \textbf{non} ti\textbf{mé}bit.

7. Parátum cor eius speráre in Dómino, † confirmátum \textbf{est} cor \textbf{e}ius: * non commovébitur donec despíciat ini\textbf{mí}cos \textbf{su}os.

8. Dispérsit, dedit paupéribus: † iustítia eius manet in \textbf{sǽ}culum \textbf{sǽ}culi, * cornu eius exaltábi\textbf{tur} in \textbf{gló}ria.

9. Peccátor vidébit, et irascétur, † déntibus suis fremet \textbf{et} ta\textbf{bé}scet: * desidérium pecca\textbf{tó}rum pe\textbf{rí}bit.

10. Glória \textbf{Pa}tri et \textbf{Fí}lio * et Spi\textbf{rí}tui \textbf{Sanc}to.

11. Sicut erat in princípio et \textbf{nunc} et \textbf{sem}per * et in sǽcula sǽcu\textbf{ló}rum. \textbf{A}men.
\end{multicols}

\vspace{0.25cm}
\gregorioscore{firstvespersant4}
\vspace{0.25cm}
\begin{multicols}{2}
2. \textit{(fit reverentia)} Sit nomen Dómini \textit{bene}\textbf{díc}tum, * ex hoc nunc, et \textit{usque in} \textbf{sǽ}culum.

3. A solis ortu usque \textit{ad oc}\textbf{cá}sum, * laudábi\textit{le nomen} \textbf{Dó}mini.

4. Excélsus super omnes \textit{gentes} \textbf{Dó}minus, * et super cælos \textit{glória} \textbf{e}ius.

5. Quis sicut Dóminus, Deus noster, qui in \textit{altis} \textbf{há}bitat, * et humília réspicit in cæ\textit{lo et in} \textbf{ter}ra?

6. Súscitans a \textit{terra} \textbf{í}nopem, * et de stércore \textit{érigens} \textbf{páu}perem:

7. Ut cóllocet eum \textit{cum prin}\textbf{cí}pibus, * cum princípibus \textit{pópuli} \textbf{su}i.

8. Qui habitáre facit stéri\textit{lem in} \textbf{do}mo, * matrem fili\textit{órum læ}\textbf{tán}tem.

9. Glória Pa\textit{tri et} \textbf{Fí}lio * et Spi\textit{rítui} \textbf{Sanc}to.

10. Sicut erat in princípio et \textit{nunc et} \textbf{sem}per * et in sǽcula sǽ\textit{culórum.} \textbf{A}men.
\end{multicols}

\vspace{0.25cm}
\gregorioscore{firstvespersant5}
\vspace{0.25cm}
\begin{multicols}{2}
2. Quóniam confirmáta est super nos misericórdia \textbf{e}ius: * et véritas Dómini manet \textbf{in} æ\textbf{tér}num.

3. Glória Patri et \textbf{Fí}lio * et Spi\textbf{rí}tui \textbf{Sanc}to.

4. Sicut erat in princípio et nunc et \textbf{sem}per * et in sǽcula sǽcu\textbf{ló}rum. \textbf{A}men.
\end{multicols}

\begin{center}
\textit{Capitulum. Cf. Es 7:27}
\end{center}

\begin{multicols}{2}
\lettrine{B}{enedíctus} Dóminus Deus patrem nostrórum † qui dedit hanc voluntátem in cor regis clarificáre domum suam * quæ est in Ierúsalem.

\begin{myfont}℟.\end{myfont} Deo grátias.
\end{multicols}

\vspace{0.25cm}
\gregorioscore{firstvespershymn}
\vspace{0.25cm}
\begin{myfont}℣.\end{myfont} David regis sedet in sólio.

\begin{myfont}℟.\end{myfont} Salomónis utens iudício.

\vspace{0.25cm}
\gregorioscore{firstvespersmagant}
\vspace{0.25cm}
\begin{multicols}{2}
3. Quia respéxit humilitátem an\textbf{cíl}læ \textbf{su}æ: * ecce enim ex hoc beátam me dicent omnes gene\textit{rati}\textbf{ó}nes.

4. Quia fecit mihi \textbf{ma}gna qui \textbf{po}tens est: * et sanctum \textit{nomen} \textbf{e}ius.

5. Et misericórdia eius, a progénie \textbf{in} pro\textbf{gé}nies: * timén\textit{tibus} \textbf{e}um.

6. Fecit poténtiam in \textbf{brác}chio \textbf{su}o: * dispérsit supérbos mente \textit{cordis} \textbf{su}i.

7. Depósuit po\textbf{tén}tes de \textbf{se}de: * et exal\textit{távit} \textbf{hú}miles.

8. Esuriéntes imp\textbf{lé}vit \textbf{bo}nis: * et dívites dimí\textit{sit in}\textbf{á}nes.

9. Suscépit Israël \textbf{pú}erum \textbf{su}um: * recordátus misericór\textit{diæ} \textbf{su}æ.

10. Sicut locútus est ad \textbf{pa}tres \textbf{no}stros: * Ábraham, et sémini ei\textit{us in} \textbf{sǽ}cula.

11. Glória \textbf{Pa}tri, et \textbf{Fí}lio, * et Spirí\textit{tui} \textbf{Sanc}to.

12. Sicut erat in princípio, et \textbf{nunc,} et \textbf{sem}per, * et in sǽcula sæcu\textit{lórum.} \textbf{A}men.
\end{multicols}

\begin{myfont}℣.\end{myfont} Dóminus vobíscum.

\begin{myfont}℟.\end{myfont} Et cum spíritu tuo.

Oremus.

\begin{center}
\textit{Oratio:}
\end{center}

\begin{multicols}{2}
\lettrine{D}{eus} qui beátum Ludovícum confessórem tuum de terréno ac temporáli regno ad cæléstis et ætérni glóriam transtulísti; † eius quǽsumus méritis et intercessióne regis regum Iesu Christi, Fílii Tui, * nos cohéredes effícias et eiúsdem regni tríbuas esse consórtes. Per Dóminum nostrum Iesum Christum, Fílium tuum: qui tecum vivit et regnat in unitáte Spíritus Sancti, Deus, per ómnia sǽcula sæculórum.

\begin{myfont}℟.\end{myfont} Amen.
\end{multicols}

\begin{myfont}℣.\end{myfont} Dóminus vobíscum.

\begin{myfont}℟.\end{myfont} Et cum spíritu tuo.

\vspace{0.25cm}
\gregorioscore{firstvespersbenedicamus}
\vspace{0.25cm}

\begin{myfont}℣.\end{myfont} Fidélium ánimæ per misericórdiam Dei requiéscant in pace.

\begin{myfont}℟.\end{myfont} Amen.

\begin{center}
\large{AD COMPLETORIVM}
\end{center}

\textit{Lector incipit}
\begin{myfont}℣.\end{myfont} Iube Domne, benedícere.

\textit{Benedictio:} Noctem quiétam et finem perféctum concédat nobis Dóminus omnípotens.

\begin{myfont}℟.\end{myfont} Amen.

\begin{center}
\textit{Lectio brevis. 1. Pet.5.c.}
\end{center}

\begin{multicols}{2}
\lettrine{F}{ratres:} sobrii estote et vigilate: † qui adversarius vester diabolus tamquam leo rugiens circuit, quærens quem devoret: * cui resistite fortes in fide. Tu autem Domine, miserere nobis.
\begin{myfont}℟.\end{myfont} Deo gratias.
\end{multicols}

\begin{myfont}℣.\end{myfont} Adiutorium nostrum in nomine Domini.

\begin{myfont}℟.\end{myfont} Qui fecit cælum et terram.

Pater noster. \textit{secreto.}

\textit{Deinde sacerdos dicit:} 

\lettrine{C}{onfiteor} Deo omnipotenti, beatæ Mariæ semper Virgini, beato Michaelo Archangelo, beato Iohanni Baptistæ, sanctis Apostolis Petro et Paulo, omnibus sanctis et vobis, fratres, quia peccavi nimis cogitatione, verbo, et opere: mea culpa, mea culpa, mea maxima culpa. Ideo precor beatam Mariam semper Virginem, beatum Michaelum Archangelum, beatum Iohannem Baptistam, sanctos Apostolos Petrum et Paulum, omnes sanctos, et vos fratres, orare pro me ad Dominum Deum nostrum.

\vspace{0.25cm}
\textit{Et chorus respondit}

\lettrine{M}{isereatur} tui omnipotens Deus, et dimissis peccatis tuis, perducat te ad vitam æternam.
\begin{myfont}℟.\end{myfont} Amen.

\vspace{0.25cm}
\textit{Et chorus repetit:}

\lettrine{C}{onfiteor} Deo omnipotenti, beatæ Mariæ semper Virgini, beato Michaelo Archangelo, beato Iohanni Baptistæ, sanctis Apostolis Petro et Paulo, omnibus sanctis et tibi, pater, quia peccavi nimis cogitatione, verbo, et opere: mea culpa, mea culpa, mea maxima culpa. Ideo precor beatam Mariam semper Virginem, beatum Michaelum Archangelum, beatum Iohannem Baptistam, sanctos Apostolos Petrum et Paulum, omnes sanctos, et te, pater, orare pro me ad Dominum Deum nostrum.

\vspace{0.25cm}
\textit{Et sacerdos respondit:}

\lettrine{M}{isereatur} vestri omnipotens Deus, et dimissis peccatis vestris, perducat vos ad vitam æternam.
\begin{myfont}℟.\end{myfont} Amen.
\vspace{0.25cm}
\lettrine{I}{ndulgentiam} absolutionem et remissionem peccatorum nostrorum tribuat vobis omnipotens et misericors Dominus.
\begin{myfont}℟.\end{myfont} Amen.

\vspace{0.5cm}
\gresetinitiallines{0}
\gregorioscore{complineconverte}
\vspace{0.25cm}
\gresetinitiallines{1}
\gregorioscore{complinepsalmsant}
\begin{multicols}{2}
2. Miserére \textbf{me}i, * et exáudi orati\textbf{ó}nem \textbf{me}am.

3. Fílii hóminum, úsquequo gravi \textbf{cor}de? * ut quid dilígitis vanitátem, et quǽri\textbf{tis} men\textbf{dá}cium?

4. Et scitóte quóniam mirificávit Dóminus sanctum \textbf{su}um: * Dóminus exáudiet me cum clamáve\textbf{ro} ad \textbf{e}um.

5. Irascímini, et nolíte peccáre: † quæ dícitis in córdibus \textbf{ves}tris, * in cubílibus vestris \textbf{com}pun\textbf{gí}mini.

6. Sacrificáte sacrifícium iustítiæ, † et speráte in \textbf{Dó}mino. * Multi dicunt: Quis osténdit \textbf{no}bis \textbf{bo}na?

7. Signátum est super nos lumen vultus tui, \textbf{Dó}mine: * dedísti lætítiam in \textbf{cor}de \textbf{me}o.

8. A fructu fruménti, vini, et ólei \textbf{su}i * mul\textbf{ti}pli\textbf{cá}ti sunt.

9. In pace in id\textbf{í}psum * dórmiam, et \textbf{re}qui\textbf{é}scam;

10. Quóniam tu, Dómine, singuláriter \textbf{in} spe * con\textbf{sti}tu\textbf{í}sti me.

11. Glória Patri et \textbf{Fí}lio * et Spi\textbf{rí}tui \textbf{Sanc}to.

12. Sicut erat in princípio et nunc et \textbf{sem}per * et in sǽcula sǽcu\textbf{ló}rum. \textbf{A}men.

\vspace{0.25cm}
\lettrine{Q}{ui} hábitat in adiutório Al\textbf{tís}simi, * in protectióne Dei cæli \textbf{com}mo\textbf{rá}bitur.

2. Dicet Dómino: Suscéptor meus es tu, et refúgium \textbf{me}um: * Deus meus spe\textbf{rá}bo in \textbf{e}um.

3. Quóniam ipse liberávit me de láqueo ve\textbf{nán}tium, * et a \textbf{ver}bo \textbf{á}spero.

4. Scápulis suis obumbrábit \textbf{ti}bi: * et sub pennis \textbf{e}ius spe\textbf{rá}bis.

5. Scuto circúmdabit te véritas \textbf{e}ius: * non timébis a ti\textbf{mó}re noc\textbf{túr}no,

6. A sagítta volánte in die, a negótio perambulánte in \textbf{té}nebris: * ab incúrsu, et dæmónio me\textbf{ri}di\textbf{á}no.

7. Cadent a látere tuo mille, et decem míllia a dextris \textbf{tu}is: * ad te autem non ap\textbf{pro}pin\textbf{quá}bit.

8. Verúmtamen óculis tuis conside\textbf{rá}bis: * et retributiónem pecca\textbf{tó}rum vi\textbf{dé}bis.

9. Quóniam tu es, Dómine, spes \textbf{me}a: * Altíssimum posuísti re\textbf{fú}gium \textbf{tu}um.

10. Non accédet ad te \textbf{ma}lum: * et flagéllum non appropinquábit taber\textbf{ná}culo \textbf{tu}o.

11. Quóniam Ángelis suis mandávit \textbf{de} te: * ut custódiant te in ómnibus \textbf{vi}is \textbf{tu}is.

12. In mánibus por\textbf{tá}bunt te: * ne forte offéndas ad lápidem \textbf{pe}dem \textbf{tu}um.

13. Super áspidem, et basilíscum ambu\textbf{lá}bis: * et conculcábis leónem \textbf{et} dra\textbf{có}nem.

14. Quóniam in me sperávit, liberábo \textbf{e}um: * prótegam eum, quóniam cognóvit \textbf{no}men \textbf{me}um.

15. Clamábit ad me, et ego exáudiam \textbf{e}um: * cum ipso sum in tribulatióne: erípiam eum et glorifi\textbf{cá}bo \textbf{e}um.

16. Longitúdine diérum replébo \textbf{e}um: * et osténdam illi salu\textbf{tá}re \textbf{me}um.

17. Glória Patri et \textbf{Fí}lio * et Spi\textbf{rí}tui \textbf{Sanc}to.

18. Sicut erat in princípio et nunc et \textbf{sem}per * et in sǽcula sǽcu\textbf{ló}rum. \textbf{A}men.

\vspace{0.25cm}
\lettrine{E}{cce} nunc benedícite \textbf{Dó}minum, * omnes \textbf{ser}vi \textbf{Dó}mini:

2. Qui statis in domo \textbf{Dó}mini, * in átriis domus \textbf{De}i \textbf{nos}tri.

3. In nóctibus extóllite manus vestras in \textbf{sanc}ta, * et bene\textbf{dí}cite \textbf{Dó}minum.

4. Benedícat te Dóminus ex \textbf{Si}on, * qui fecit \textbf{cæ}lum et \textbf{ter}ram.

5. Glória Patri, et \textbf{Fí}lio, * et Spi\textbf{rí}tui \textbf{Sanc}to.

6. Sicut erat in princípio et nunc et \textbf{sem}per * et in sǽcula sǽcu\textbf{ló}rum. \textbf{A}men.
\end{multicols}
\gregorioscore{complinepsalmsant}
\vspace{0.25cm}
[Hymnus \textit{Te lucis ante terminum}]

\begin{center}
\textit{Capitulum Ier. 14.}
\end{center}
\begin{multicols}{2}
\lettrine{T}{u} autem in nobis es Dómine: † et nomen sanctum tuum invocátum est super nos: * ne derelínquas nos Dómine Deus noster.

\begin{myfont}℟.\end{myfont} Deo grátias.
\end{multicols}

\gregorioscore{complineshortresp}
\vspace{0.25cm}
\begin{myfont}℣.\end{myfont} Custódi nos Dómine ut pupíllam óculi.

\begin{myfont}℟.\end{myfont} Sub umbra alárum tuárum prótege nos.
\vspace{0.25cm}
\gregorioscore{complinenuncdimittis}
\begin{multicols}{2}
2. Quia vidérunt \textbf{ó}culi \textbf{me}i * salu\textit{táre} \textbf{tu}um:

3. \textbf{Quod} pa\textbf{rá}sti * ante fáciem ómnium \textit{popu}\textbf{ló}rum:

4. Lumen ad revelati\textbf{ó}nem \textbf{gén}tium, * et glóriam plebis \textit{tuæ} \textbf{Ís}rael.

5. Glória \textbf{Pa}tri, et \textbf{Fí}lio, * et Spirí\textit{tui} \textbf{Sanc}to.

6. Sicut erat in princípio, et \textbf{nunc,} et \textbf{sem}per, * et in sǽcula sǽcu\textit{lórum.} \textbf{A}men.
\end{multicols}

\begin{myfont}℣.\end{myfont} Dóminus vobíscum.

\begin{myfont}℟.\end{myfont} Et cum spíritu tuo.

Oremus.

\begin{center}
\textit{Oratio:}
\end{center}

\begin{multicols}{2}
\lettrine{V}{ísita,} quǽsumus, Dómine, habitatiónem istam, et omnes insídias inimíci ab ea lónge repélle: † Ángeli tui sancti hábitent in ea, qui nos in pace custódiant; * et benedíctio tua sit super nos semper. Per Dóminum nostrum Iesum Christum, Fílium tuum: qui tecum vivit et regnat in unitáte Spíritus Sancti, Deus, per ómnia sǽcula sæculórum.

\begin{myfont}℟.\end{myfont} Amen.
\end{multicols}
\begin{myfont}℣.\end{myfont} Dóminus vobíscum.

\begin{myfont}℟.\end{myfont} Et cum spirítu tuo.

\begin{myfont}℣.\end{myfont} Benedicámus Dómino.

\begin{myfont}℟.\end{myfont} Deo grátias.

\textit{Benedictio:} Benedícat et custódiat nos omnípotens et miséricors Dóminus, Pater, et Fílius, et Spíritus Sanctus.

\begin{myfont}℟.\end{myfont} Amen.

\begin{center}
\large{AD MATVTINVM}
\end{center}

\vspace{0.25cm}
Pater noster. Ave Maria. Credo.
\gregorioscore{matinsdominelabia}
\gregorioscore{matinsinvitatory}
\gregorioscore{matinshymn}

\begin{center}
In I Nocturno
\end{center}
\gregorioscore{matinsant1}
\begin{multicols}{2}
2. Sed in lege Dómini volúntas ejus, * et in lege ejus meditábitur die ac nocte.

3. Et erit tamquam lignum, quod plantátum est secus decúrsus aquárum, * quod fructum suum dabit in témpore suo:

4. Et fólium ejus non défluet: * et ómnia quæcúmque fáciet, prosperabúntur.

5. Non sic ímpii, non sic: * sed tamquam pulvis, quem próicit ventus a fácie terræ.

6. Ídeo non resúrgent ímpii in judício: * neque peccatóres in concílio justórum.

7. Quóniam novit Dóminus viam justórum: * et iter impiórum períbit.

8. Glória Patri, et Fílio, * et Spirítui Sancto.

9. Sicut erat in princípio et nunc et semper: * et in sǽcula sæculórum. Amen.
\end{multicols}
\gregorioscore{matinsant2}
\begin{multicols}{2}
2. Astitérunt reges terræ, et príncipes convenérunt in unum * advérsus Dóminum, et advérsus Christum ejus.

3.  Dirumpámus víncula eórum: * et proiciámus a nobis jugum ipsórum.

4. Qui hábitat in cælis, irridébit eos: * et Dóminus subsannábit eos.

5. Tunc loquétur ad eos in ira sua, * et in furóre suo conturbábit eos.

6. Ego autem constitútus sum Rex ab eo super Sion montem sanctum ejus, * prǽdicans præcéptum ejus.

7. Dóminus dixit ad me: * Fílius meus es tu, ego hódie génui te.

8. Póstula a me, et dabo tibi gentes hereditátem tuam, * et possessiónem tuam términos terræ.

9. Reges eos in virga férrea, * et tamquam vas fíguli confrínges eos.

10. Et nunc, reges, intellégite: * erudímini, qui judicátis terram.

11. Servíte Dómino in timóre: * et exsultáte ei cum tremóre.

12. Apprehéndite disciplínam, nequándo irascátur Dóminus, * et pereátis de via justa.

13. Cum exárserit in brevi ira ejus: * beáti omnes qui confídunt in eo.

14. Glória Patri, et Fílio, * et Spirítui Sancto.

15. Sicut erat in princípio et nunc et semper: * et in sǽcula sæculórum. Amen.
\end{multicols}
\gregorioscore{matinsant3}
\begin{multicols}{2}
2. Multi dicunt ánimæ meæ: * Non est salus ipsi in Deo ejus.

3. Tu autem, Dómine, suscéptor meus es, * glória mea, et exáltans caput meum.

4. Voce mea ad Dóminum clamávi: * et exaudívit me de monte sancto suo.

5. Ego dormívi, et soporátus sum: * et exsurréxi, quia Dóminus suscépit me.

6. Non timébo míllia pópuli circumdántis me: * exsúrge, Dómine, salvum me fac, Deus meus.

7. Quóniam tu percussísti omnes adversántes mihi sine causa: * dentes peccatórum contrivísti.

8. Dómini est salus: * et super pópulum tuum benedíctio tua.

9. Glória Patri, et Fílio, * et Spirítui Sancto.

10. Sicut erat in princípio et nunc et semper: * et in sǽcula sæculórum. Amen.
\end{multicols}

\vspace{0.25cm}
\begin{myfont}℣.\end{myfont} Amávit eum Dóminus, et ornávit eum.

\begin{myfont}℟.\end{myfont} Stolam glóriæ índuit eum.

Pater noster \textit{secreto usque ad:}

\begin{myfont}℣.\end{myfont} Et ne nos indúcas in tentatiónem.

\begin{myfont}℟.\end{myfont} Sed líbera nos a malo.

\textit{Absolutio:} Exáudi, Dómine Iesu Christe, † preces servórum tuórum, et miserére nobis: * Qui cum Patre et Spíritu Sancto vivis et regnas in sǽcula sæculórum. Amen.

\begin{myfont}℣.\end{myfont} Iube, domne, benedícere.

\textit{Benedictio:} Benedictione perpétua * benedicat nos Pater ætérnus.

\begin{myfont}℟.\end{myfont} Amen.

\begin{center}
De Libro Ecclesiastici \textit{Sir. 31:8-11}
\end{center}
\begin{multicols}{2}
\lettrine{B}{eátus} vir, qui invéntus est sine mácula, et qui post aurum non ábiit, nec sperávit in pecúnia et thesáuris. Quis est hic et laudábimus eum? Fecit enim mirabília in vita sua. Qui probátus est in illo, et perféctus est, erit illi glória ætérna: Qui pótuit tránsgredi, et non est transgréssus; fácere mala, et non fecit: ídeo stabilíta sunt bona illíus in Dómino, et eleemósynas illíus enarrábit omnis ecclésia sanctórum.
\end{multicols}

\begin{myfont}℣.\end{myfont} Tu autem, Dómine, miserére nobis.

\begin{myfont}℟.\end{myfont} Deo grátias.
\gregorioscore{matinsresp1}

\begin{center}
\large{AD LAVDES}
\end{center}

\vspace{0.25cm}
Deus in adiutorium.

\gregorioscore{laudsant1}
\begin{multicols}{2}
2. Étenim firmávit orbem terræ, * qui non commovébitur.

3. Paráta sedes tua ex tunc: * a sǽculo tu es.

4. Elevavérunt flúmina, Dómine: * elevavérunt flúmina vocem suam.

5. Elevavérunt flúmina fluctus suos, * a vócibus aquárum multárum.

6. Mirábiles elatiónes maris: * mirábilis in altis Dóminus.

7. Testimónia tua credibília facta sunt nimis: * domum tuam decet sanctitúdo, Dómine, in longitúdinem diérum.

8. Glória Patri, et Fílio, * et Spirítui Sancto.

9. Sicut erat in princípio et nunc et semper * et in sǽcula sǽculórum. Amen.
\end{multicols}
\vspace{0.25cm}
\gregorioscore{laudsant2}
\begin{multicols}{2}
2. Introíte in conspéctu eius, * in exsultatióne.

3. Scitóte quóniam Dóminus ipse est Deus: * ipse fecit nos, et non ipsi nos.

4. Pópulus eius, et oves páscuæ eius: * introíte portas eius in confessióne, átria eius in hymnis: confitémini illi.

5. Laudáte nomen eius: quóniam suávis est Dóminus, in ætérnum misericórdia eius, * et usque in generatiónem et generatiónem véritas eius.

6. Glória Patri, et Fílio, * et Spirítui Sancto.

7. Sicut erat in princípio et nunc et semper * et in sǽcula sǽculórum. Amen.
\end{multicols}
\vspace{0.25cm}
\gregorioscore{laudsant3}
\begin{multicols}{2}
2. Sitívit in te ánima mea, * quam multiplíciter tibi caro mea.

3. In terra desérta, et ínvia, et inaquósa: * sic in sancto appárui tibi, ut vidérem virtútem tuam, et glóriam tuam.

4. Quóniam mélior est misericórdia tua super vitas: * lábia mea laudábunt te.

5. Sic benedícam te in vita mea: * et in nómine tuo levábo manus meas.

6. Sicut ádipe et pinguédine repleátur ánima mea: * et lábiis exsultatiónis laudábit os meum.

7. Si memor fui tui super stratum meum, in matutínis meditábor in te: * quia fuísti adiútor meus.

8. Et in velaménto alárum tuárum exsultábo, adhǽsit ánima mea post te: * me suscépit déxtera tua.

9. Ipsi vero in vanum quæsiérunt ánimam meam, introíbunt in inferióra terræ: * tradéntur in manus gládii, partes vúlpium erunt.

10. Rex vero lætábitur in Deo, laudabúntur omnes qui iurant in eo: * quia obstrúctum est os loquéntium iníqua.

\lettrine{D}{eus} misereátur nostri, et benedícat nobis: * illúminet vultum suum super nos, et misereátur nostri.

2. Ut cognoscámus in terra viam tuam, * in ómnibus géntibus salutáre tuum.

3. Confiteántur tibi pópuli, Deus: * confiteántur tibi pópuli omnes.

4. Læténtur et exsúltent gentes: * quóniam iúdicas pópulos in æquitáte, et gentes in terra dírigis.

5. Confiteántur tibi pópuli, Deus, confiteántur tibi pópuli omnes: * terra dedit fructum suum.

6. Benedícat nos Deus, Deus noster, benedícat nos Deus: * et métuant eum omnes fines terræ.

7. Glória Patri, et Fílio, * et Spirítui Sancto.

9. Sicut erat in princípio et nunc et semper * et in sǽcula sǽculórum. Amen.
\end{multicols}
\vspace{0.25cm}
\gregorioscore{laudsant4}
\begin{multicols}{2}
2. Benedícite, Ángeli Dómini, Dómino: * benedícite, cæli, Dómino.

3. Benedícite, aquæ omnes, quæ super cælos sunt, Dómino: * benedícite, omnes virtútes Dómini, Dómino.

4. Benedícite, sol et luna, Dómino: * benedícite, stellæ cæli, Dómino.

5. Benedícite, omnis imber et ros, Dómino: * benedícite, omnes spíritus Dei, Dómino.

6. Benedícite, ignis et æstus, Dómino: * benedícite, frigus et æstus, Dómino.

7. Benedícite, rores et pruína, Dómino: * benedícite, gelu et frigus, Dómino.

8. Benedícite, glácies et nives, Dómino: * benedícite, noctes et dies, Dómino.

9. Benedícite, lux et ténebræ, Dómino: * benedícite, fúlgura et nubes, Dómino.

10. Benedícat terra Dóminum: * laudet et superexáltet eum in sǽcula.

11. Benedícite, montes et colles, Dómino: * benedícite, univérsa germinántia in terra, Dómino.

12. Benedícite, fontes, Dómino: * benedícite, mária et flúmina, Dómino.

13. Benedícite, cete, et ómnia, quæ movéntur in aquis, Dómino: * benedícite, omnes vólucres cæli, Dómino.

14. Benedícite, omnes béstiæ et pécora, Dómino: * benedícite, fílii hóminum, Dómino.

15. Benedícat Israël Dóminum: * laudet et superexáltet eum in sǽcula.

16. Benedícite, sacerdótes Dómini, Dómino: * benedícite, servi Dómini, Dómino.

17. Benedícite, spíritus, et ánimæ iustórum, Dómino: * benedícite, sancti, et húmiles corde, Dómino.

18. Benedícite, Ananía, Azaría, Mísaël, Dómino: * laudáte et superexaltáte eum in sǽcula.

19. \textit{(fit reverentia:)} Benedicámus Patrem et Fílium cum Sancto Spíritu: * laudémus et superexaltémus eum in sǽcula.

20. Benedíctus es, Dómine, in firmaménto cæli: * et laudábilis, et gloriósus, et superexaltátus in sǽcula.
\end{multicols}
\vspace{0.25cm}
\gregorioscore{laudsant5}
\begin{multicols}{2}
2. Laudáte eum, omnes Ángeli eius: * laudáte eum, omnes virtútes eius.

3.Laudáte eum, sol et luna: * laudáte eum, omnes stellæ et lumen.

4. Laudáte eum, cæli cælórum: * et aquæ omnes, quæ super cælos sunt, laudent nomen Dómini.

5. Quia ipse dixit, et facta sunt: * ipse mandávit, et creáta sunt.

6. Státuit ea in ætérnum, et in sǽculum sǽculi: * præcéptum pósuit, et non præteríbit.

7. Laudáte Dóminum de terra, * dracónes, et omnes abýssi.

8. Ignis, grando, nix, glácies, spíritus procellárum: * quæ fáciunt verbum eius:

9. Montes, et omnes colles: * ligna fructífera, et omnes cedri.

10. Béstiæ, et univérsa pécora: * serpéntes, et vólucres pennátæ:

11. Reges terræ, et omnes pópuli: * príncipes, et omnes iúdices terræ.

12. iúvenes, et vírgines: † senes cum iunióribus laudent nomen Dómini: * quia exaltátum est nomen eius solíus.

13. Conféssio eius super cælum et terram: * et exaltávit cornu pópuli sui.

14. Hymnus ómnibus sanctis eius: * fíliis Israël, pópulo appropinquánti sibi.

\lettrine{C}{antáte} Dómino cánticum novum: * laus eius in ecclésia sanctórum.

2. Lætétur Israël in eo, qui fecit eum: * et fílii Sion exsúltent in rege suo.

3. Laudent nomen eius in choro: * in týmpano, et psaltério psallant ei:

4. Quia beneplácitum est Dómino in pópulo suo: * et exaltábit mansuétos in salútem.

5. Exsultábunt sancti in glória: * lætabúntur in cubílibus suis.

6. Exaltatiónes Dei in gútture eórum: * et gládii ancípites in mánibus eórum.

7. Ad faciéndam vindíctam in natiónibus: * increpatiónes in pópulis.

8. Ad alligándos reges eórum in compédibus: * et nóbiles eórum in mánicis férreis.

9. Ut fáciant in eis iudícium conscríptum: * glória hæc est ómnibus sanctis eius.

\lettrine{L}{audáte} Dóminum in sanctis eius: * laudáte eum in firmaménto virtútis eius.

2. Laudáte eum in virtútibus eius: * laudáte eum secúndum multitúdinem magnitúdinis eius.

3 .Laudáte eum in sono tubæ: * laudáte eum in psaltério, et cíthara.

4. Laudáte eum in týmpano, et choro: * laudáte eum in chordis, et órgano.

5. Laudáte eum in cýmbalis benesonántibus: † laudáte eum in cýmbalis iubilatiónis: * (6) omnis spíritus laudet Dóminum.

6. Glória Patri, et Fílio, * et Spirítui Sancto.

7. Sicut erat in princípio et nunc et semper * et in sǽcula sǽculórum. Amen.
\end{multicols}

\begin{center}\textit{Capitulum}\end{center}
\begin{multicols}{2}
\lettrine{D}{edit} Dominus illi fortitudinem et usque ad senectutem permansit illi virtus, ut ascenderet in excelsum terræ locum, semen ipsius obtinebit hereditatem. \begin{myfont}℟.\end{myfont} Deo gratias.
\end{multicols}
\gregorioscore{laudshymn}
\vspace{0.25cm}

\begin{myfont}℣.\end{myfont} Versiculus.

\begin{myfont}℟.\end{myfont} Responsorium.

\gregorioscore{laudsbenedictusant}
\begin{multicols}{2}
2. Et eréxit cornu salútis nobis: * in domo David, púeri sui.

3. Sicut locútus est per os sanctórum, * qui a sǽculo sunt, prophetárum eius:

4. Salútem ex inimícis nostris, * et de manu ómnium, qui odérunt nos.

5. Ad faciéndam misericórdiam cum pátribus nostris: * et memorári testaménti sui sancti.

6. iusiurándum, quod iurávit ad Ábraham patrem nostrum, * datúrum se nobis:

7. Ut sine timóre, de manu inimicórum nostrórum liberáti, * serviámus illi.

8. In sanctitáte, et iustítia coram ipso, * ómnibus diébus nostris.

9. Et tu, puer, Prophéta Altíssimi vocáberis: * præíbis enim ante fáciem Dómini, paráre vias eius:

10. Ad dandam sciéntiam salútis plebi eius: * in remissiónem peccatórum eórum:

11. Per víscera misericórdiæ Dei nostri: * in quibus visitávit nos, óriens ex alto:

12. Illumináre his, qui in ténebris, et in umbra mortis sedent: * ad dirigéndos pedes nostros in viam pacis.

13. Glória Patri, et Fílio, * et Spirítui Sancto.

14. Sicut erat in princípio et nunc et semper * et in sǽcula sǽculórum. Amen.
\end{multicols}

\begin{myfont}℣.\end{myfont} Dominus vobiscum.

\begin{myfont}℟.\end{myfont} Et cum spiritu tuo.

Oremus.

\begin{center}
\textit{Oratio:}
\end{center}

\begin{multicols}{2}
\lettrine{D}{eus} qui beatum Ludovicum confessorem tuum de terreno ac temporali regno ad caelestis et aeterni gloriam transtulisti; † eius quæsumus meritis et intercessione regis regum iesu Christi, Filii Tui, * nos coheredes efficias et eiusdem regni tribuas esse consortes. Per Dóminum nostrum Iesum Christum, Fílium tuum: qui tecum vivit et regnat in unitáte Spíritus Sancti, Deus, per ómnia sǽcula sæculórum.

\begin{myfont}℟.\end{myfont} Amen.
\end{multicols}

\begin{myfont}℣.\end{myfont} Dominus vobiscum.

\begin{myfont}℟.\end{myfont} Et cum spiritu tuo.

\vspace{0.25cm}
\gregorioscore{firstvespersbenedicamus}
\vspace{0.25cm}

\begin{myfont}℣.\end{myfont} Fidelium animæ per misericordiam Dei requiescant in pace.

\begin{myfont}℟.\end{myfont} Amen.

\begin{center}\large{AD PRIMAM}\end{center}

Pater noster. Ave Maria. Et reliqua.

\begin{center}\large{AD TERTIAM}\end{center}

Pater noster. Ave Maria. Et reliqua.

\begin{center}\large{AD MISSAM}\end{center}

Textus gabc de Vincentio Regazzi.

\begin{center}\large{AD SEXTAM}\end{center}

Pater noster. Ave Maria. Et reliqua.

\begin{center}\large{AD NONAM}\end{center}

Pater noster. Ave Maria. Et reliqua.

\begin{center}\large{AD II. VESPERAS}\end{center}

Pater noster. Ave Maria. Et reliqua.

\end{document}